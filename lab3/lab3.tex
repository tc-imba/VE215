\documentclass{article}
\usepackage{enumerate}
\usepackage{amsmath}
\usepackage{amssymb}
\usepackage{graphicx}
\usepackage{subfigure}
\usepackage{geometry}
\usepackage{color}
\usepackage{bm}
\usepackage{indentfirst}
\usepackage{array}
\usepackage{multirow}
\usepackage{diagbox}

\begin{document}

\vspace*{0.25cm}

\hrulefill

\thispagestyle{empty}

\begin{center}
\begin{large}
\sc{UM--SJTU Joint Institute \vspace{0.3em} \\ Physics Laboratory \\(VE215)}
\end{large}

\hrulefill

\vspace*{5cm}
\begin{Large}
\sc{{Laboratory Report}}
\end{Large}

\vspace{2em}

\begin{large}
\sc{{Exercise 3
\vspace{0.5em}

Transient Lab
}}
\end{large}
\end{center}


\vfill

\begin{table}[h!]
\flushleft
\begin{tabular}{lll}
Name: Yihao Liu \hspace*{2em}&
ID: 515370910207\hspace*{2em}\\


\\

Date: 27 Oct 2016 

\end{tabular}
\end{table}

\hfill
\begin{tiny}
[rev. 1.0]
\end{tiny}
\newpage


\section{Goal}
\begin{enumerate}
\item
Apply the theory you learned on the step responses in rst- and second-
order circuits to series RC and RLC circuits, which you will build in the lab.
\item
Build a series RC circuit, observe its responses to input square wave
signal of varied frequency, and explain them based on the theory you learned:
\begin{enumerate}[$\bullet$]
\item
Relate the observed capacitor voltage and resistor voltage as functions of
time to your pre-lab calculations
\item
Explain the changes of both output waveforms in response to the increase
of the frequency of the input square wave signal
\item
Explain the amplitudes of the capacitor voltage and the resistor voltage
related to the amplitude of the input square wave
\end{enumerate}
\item
Build a series RLC circuit, observe the three types of its responses to input
square wave signal, and relate them to the theory you have learned. For the
under-damped/ over-damped/ critical damped response, compare the resistance
in the circuit measured in the lab with the critical resistance you calculated in
the pre-lab.
\item
Build the simplest second-order circuit, an LC tank, and observe oscilla-
tions.
\end{enumerate}

\section{Introduction}

\subsection{First-order circuits}
Theoretically, the transient responses in electric circuits are described by
differential equations. The circuits, whose responses obey the first-order differential equation

$$\frac{dx(t)}{dt}+\frac{1}{\tau}\cdot x(t)=f(t)$$

are called first-order circuits. Their responses are always monotonic and
appear in the form of exponential function

$$x(t)=K_1\cdot e^{-\frac{t}{\tau}}+K_2$$

A first-order circuit includes the effective resistance R and one energy-storage
element, an inductor L or a capacitor C.

In an RC circuit, the time constant is

$$\tau=RC$$

In an LC circuit, the time constant is

$$\tau=\frac{L}{R}$$

The fall time of a signal is defined as the interval between the moment when the signal reaches its 90\% and the moment when the signal reaches its 10\% level. Note that the 10\% level is reached between 2$\tau$ and 3$\tau$ . Approximately, you can assume falltime $\approx2.2\tau$ . After $t = 5\tau$ , the exponent practically equals zero.

\subsection{Second-order circuits}


Many circuits involve two energy-storing elements, both an inductor L and a
capacitor C. Such circuits require a second-order differential equation description

$$\frac{d^2x(t)}{dt^2}+2\cdot\alpha\cdot\frac{dx(t)}{dt}+\omega_0^2\cdot x(t)=f(t)$$

thus they are called second-order circuits.

We will consider only second-order circuits with one inductor and one capacitor. The differential equation includes two parameters: the damping factor $\alpha$ and the undamped frequency $\omega_0$ which are determined by the circuit and its components.

For example, in the series RLC circuit, which you will build and study in this lab,

$$\alpha=\frac{R}{2\cdot L},\rm{and}\ \omega_0=\frac{1}{\sqrt{L\cdot C}}$$

while in the parallel RLC circuit,

$$\alpha=\frac{1}{2\cdot R\cdot C},\rm{and}\ \omega_0=\frac{1}{\sqrt{L\cdot C}}$$

Depending on the two parameters $\alpha$ and $\omega_0$, second-order circuits can exhibit three types of responses.

\subsubsection{The underdamped response}

If $\alpha<\omega_0$

$$x(t)=e^{-\alpha t}(K_1\cos(\omega t)+K_2\sin(\omega t))$$

where $\omega=\sqrt{\omega_0^2-\alpha^2}$

The underdamped circuit response involves decaying oscillations, which may
last for many periods or for less than one period, depending on the damping
ratio $\delta=\frac{\alpha}{\omega_0}$, which for the series RLC circuit 
$\delta=\frac{R}{2L}\sqrt{LC}=\frac{R}{2}\cdot\sqrt{\frac{C}{L}}$. Varying 
the values of R, L, C, affects the damping ratio $\delta$.

\subsubsection{The critically damped response}

If $\alpha=\omega_0$

$$x(t)=e^{-\alpha t}(K_1+K_2t)$$

and the circuit has the critically damped response.

The critically damped response does not involve oscillations.

For the series RLC circuits, $\alpha=\omega_0$ corresponds to $\frac{R}{2L}=\frac{1}{\sqrt{LC}}$ or $R=R_{critical}=2\sqrt{\frac{L}{C}}$

If $L = 1mH$ and $C = 10nF$, then $R_{critical} \approx 632 \Omega$.

\subsubsection{The overdamped response}

If $\alpha>\omega_0$

$$x(t)=K_1\cdot e^{s_1t}+K_2\cdot e^{s_2t}$$

where $s_1=-\alpha+\sqrt{\alpha^2-\omega_0^2}$ and $s_2=-\alpha-\sqrt{\alpha^2-\omega_0^2}$

In the series RLC circuits, the overdamped solution is obtained if the resistance is larger that the critical resistance, such that $R>R_{critical}=2\sqrt{\frac{L}{C}}$

Notice that the larger resistance corresponds to the longer delay, and even
the faster decay has a much longer fall time than the critically damped response.

One of the most interesting features of series RLC circuits is that increasing
the resistance above the critical value results in much longer fall time, or longer
delays of responses in digital circuits. Among all monotonic responses, the
critically damped is the fastest.

\section{Results and Discussion}

\subsection{First-order circuits}

\begin{table}[!h]
\begin{center}
\begin{tabular}{|c|c|c|}
\hline
The setting of the potentiometer  & Fastest circuit response & Slowest circuit response \\
\hline
$V_{ppk}$ input [V]		&	1.13	&	1.13	\\
\hline
$V_{ppk}$ output [V]	&	1.13	&	1.13	\\
\hline
Period $T$ of the input [ms]	&	10.000	&	10.000	\\
\hline
Rise Time of the Output [ms]	&	0.230	&	3.400	\\
\hline
Fall Time of the Output [ms]	&	0.246	&	3.040	\\
\hline
\end{tabular}
\caption{First-order circuits.}
\label{tab-1}
\end{center}
\end{table}

For the fastest circuit response,

$$\tau=\frac{0.230+0.246}{2}\cdot\frac{1}{2.2}=0.108\,\rm{ms}$$

Theoretically,

$$\tau=RC=0.100\,\rm{ms}$$

The relative error is $8\%$\\

For the slowest circuit response,

$$\tau=\frac{3.400+3.040}{2}\cdot\frac{1}{2.2}=1.464\,\rm{ms}$$

Theoretically,

$$\tau=RC=1.100\,\rm{ms}$$

The relative error is $33.1\%$\\

We can find that in the slowest circuit response, the relative error is large. It is probably because if the error of the resistance and the oscilloscope.

\subsection{Second-order circuits}

\begin{table}[!h]
\begin{center}
\begin{tabular}{|c|c|c|c|c|}
\hline
& Resistance $R_p$ [$k\Omega$] & Rise Time [$ms$] & Fall Time [$ms$] & Time interval $\Delta t$ [$\mu s$] \\
\hline
\multirow{2}{*}{Under-damped}		&	0.000	&	0.00165	&	0.0016	&	5.3	\\
\cline{2-5}
&	0.532	&	0.0018	&	0.0017	&	5.8	\\
\hline
Critically damped	&	2.100	&	0.0029	&	0.0029	&	\multirow{3}{*}{\diagbox[height=3.5em,width=11em,dir=SE]{}{}}\\
\cline{1-4}
\multirow{2}{*}{Over-damped}	&	4.460	&	0.0076	&	0.0078	&	\\
\cline{2-4}
&	10.000	&	0.0176	&	0.0172	&	\\
\hline
\end{tabular}
\caption{Second-order circuits.}
\label{tab-2}
\end{center}
\end{table}

For the first under-damped circuit,

$$\omega=\frac{2\pi}{\Delta t}=1.186\times10^9$$

Theoretically,

$$\omega=\sqrt{\omega_0^2-\alpha^2}=\sqrt{\frac{1}{LC}-\frac{R^2}{4L^2}}=1.103\times10^9$$

The relative error is $7.5\%$

$$\delta=\frac{R}{2}\sqrt{\frac{C}{L}}=0.045$$\\

For the second under-damped circuit,

$$\omega=\frac{2\pi}{\Delta t}=1.083\times10^9$$

Theoretically,

$$\omega=\sqrt{\omega_0^2-\alpha^2}=\sqrt{\frac{1}{LC}-\frac{R^2}{4L^2}}=1.058\times10^9$$

The relative error is $2.4\%$\\

$$\delta=\frac{R}{2}\sqrt{\frac{C}{L}}=0.286$$\\

Since $\delta_1<\delta_2$, the damping in the first circuit decays more slowly.\\

For the critically damped circuit,

$$R_{critical}=2\sqrt{\frac{L}{C}}=2208.63\Omega$$


\section{Conclusion}
In the lab, we learned how a build a RC and RLC circuit. We also found the properties if these circuits, especially the relationship between R,L,C and the time period constant $\tau$. We discovered $\delta$ in an under-damped RLC circuit to find how fast it decays.

\section{Reference}
Lab 3 Manual.

\section{Pre-lab and Data sheet}

\end{document}